%-------------------------
% Resume in Latex
% Author : Novichkov Sergey
% License : MIT
%------------------------

\documentclass[letterpaper,11pt]{article}

\usepackage{latexsym}
\usepackage[empty]{fullpage}
\usepackage{titlesec}
\usepackage{marvosym}
\usepackage[usenames,dvipsnames]{color}
\usepackage{verbatim}
\usepackage{enumitem}
\usepackage[hidelinks]{hyperref}
\usepackage{fancyhdr}
\usepackage[english, russian]{babel}
\usepackage{tabularx}
\input{glyphtounicode}


%----------FONT OPTIONS----------
% sans-serif
% \usepackage[sfdefault]{FiraSans}
% \usepackage[sfdefault]{roboto}
% \usepackage[sfdefault]{noto-sans}
% \usepackage[default]{sourcesanspro}

% serif
% \usepackage{CormorantGaramond}
% \usepackage{charter}

\setlength{\parskip}{0pt} % Убирает расстояние между абзацами
\setlength{\parindent}{0pt} % Убирает отступ первой строки абзаца

\pagestyle{fancy}
\fancyhf{} % clear all header and footer fields
\fancyfoot{}
\renewcommand{\headrulewidth}{0pt}
\renewcommand{\footrulewidth}{0pt}

% Adjust margins

\documentclass{article}
\usepackage[left=0.5in, right=0.5in, top=0.3in, bottom=0.5in]{geometry}

\urlstyle{same}

\raggedbottom
\raggedright
\setlength{\tabcolsep}{0in}

% Sections formatting
\titleformat{\section}{
  \vspace{-1pt}\scshape\raggedright\normalsize
}{}{0em}{}[\color{black}\titlerule \vspace{-1pt}]

% Ensure that generate pdf is machine readable/ATS parsable
\pdfgentounicode=1

%-------------------------
% Custom commands
\newcommand{\resumeItem}[1]{
  \item\small{
    {#1 \vspace{-20pt}}
  }
}

\newcommand{\resumeSubheading}[4]{
  \vspace{-2pt}\item
    \begin{tabular*}{0.97\textwidth}[t]{l@{\extracolsep{\fill}}r}
      \textbf{#1} & #2 \\
      \textit{\small#3} & \textit{\small #4} \\
    \end{tabular*}\vspace{-7pt}
}

\newcommand{\resumeSubSubheading}[2]{
    \item
    \begin{tabular*}{0.97\textwidth}{l@{\extracolsep{\fill}}r}
      \textit{\small#1} & \textit{\small #2} \\
    \end{tabular*}\vspace{-7pt}
}

\newcommand{\resumeProjectHeading}[2]{
    \item
    \begin{tabular*}{0.97\textwidth}{l@{\extracolsep{\fill}}r}
      \small#1 & #2 \\
    \end{tabular*}\vspace{-7pt}
}

\newcommand{\resumeSubItem}[1]{\resumeItem{#1}\vspace{-4pt}}

\renewcommand\labelitemii{$\vcenter{\hbox{\tiny$\bullet$}}$}

\newcommand{\resumeSubHeadingListStart}{\begin{itemize}[leftmargin=0.15in, label={}]}
\newcommand{\resumeSubHeadingListEnd}{\end{itemize}}
\newcommand{\resumeItemListStart}{\begin{itemize}\vspace{25pt}}
\newcommand{\resumeItemListEnd}{\end{itemize}\vspace{-5pt}}

%-------------------------------------------
%%%%%%  RESUME STARTS HERE  %%%%%%%%%%%%%%%%%%%%%%%%%%%%

\begin{document}

\begin{center}
    \textbf{\huge \scshape Новичков Сергей} \\ \vspace{1pt}
    \textbf{\scshape Senior ML Engineer | Team Lead | Product Manager} \\
    \href{mailto:novichkovSD@ya.ru}{\uline{novichkovSD@ya.ru}} $|$ \href{https://t.me/xEnotWhyNotx}{\uline{@xEnotWhyNotx}} $|$ \href{https://linkedin.com/in/xEnotWhyNotx}{\uline{LinkedIn}} $|$ \href{https://github.com/xEnotWhyNotx}{\uline{GitHub}}
    \vspace{-20pt}
\end{center}

%-----------SELF PROMO-----------
\section{Обо мне}
\begin{itemize}[leftmargin=0.15in, label={}]
    \item \vspace{-5pt} 
    Senior ML Engineer и Team Lead с 3+ годами опыта в СБЕР и финтех-стартапах. Руковожу командой разработки ML-решений, которые напрямую влияют на доходы компании через алгоритмы многорукого бандита для динамического изменения ключевой ставки по факторингу. Архитектор AI-агентов с RAG-системами (GigaChat2, Qwen3) для внутренних сервисов автоматизации. Лидер внедрения MCP-серверов в корпоративной среде. Успешно трансформирую бизнес-требования в масштабируемые ML-продукты, снизил операционные расходы на 35 млн руб./год. Эксперт в продуктовой стратегии, экспериментировании и кросс-функциональном лидерстве.
    \vspace{-15pt}
\end{itemize}

%-----------PROGRAMMING SKILLS-----------
\section{Навыки}
\begin{itemize}[leftmargin=0.15in, label={}]
    \small{\item{
        \vspace{-5pt}
        \textbf{Лидерство \& Продукт}: Управление командами 5+ человек, продуктовые стратегии, OKR, A/B тестирование, кросс-функциональная координация, PRD, экспериментирование \\
        \textbf{ML \& AI}: LLMs (GigaChat2, Qwen3), RAG-системы, LangChain, FAISS, многорукие бандиты, XGBoost, CatBoost, Transformers, SpaCy, NER \\
        \textbf{Технологии}: Python, SQL, Docker, Kubernetes, RabbitMQ, Kafka, Hadoop, GitLab CI/CD, Grafana, Streamlit, Flask \\
        \textbf{Инновации}: MCP-серверы, AI-агенты, микросервисная архитектура, Big Data ETL, PII Detection \\
        \textbf{Языки}: Русский (родной), Английский (B2)
        \vspace{-15pt}
    }}
\end{itemize}

%-----------EXPERIENCE-----------
\section{Опыт работы}

\begin{itemize}[leftmargin=0.15in, label={}]
    \item \textbf{Senior ML Engineer / Team Lead} \hfill \textit{03.2024 -- настоящее время} \\
    \textit{\textbf{СБЕР}, Москва, Россия} \\
    \vspace{-10pt}
    \begin{itemize}
        \item Руковожу командой ML-разработки, влияющей на доходы через алгоритмы многорукого бандита для динамического изменения ключевой ставки по факторингу
        \item Разработал и внедрил сервис агента для помощи сотрудникам с несколькими AI-ассистентами на базе RAG-систем с GigaChat2 и Qwen3, снизил HR-тикеты на 60\%
        \item Лидер внедрения MCP-серверов в корпоративной среде, веду переговоры с отделами по внедрению GenAI-решений
        \item Автоматизировал 70\% документирования корпоративных таблиц через LLM-пайплайн с RabbitMQ, сэкономил компании 35 млн руб./год
        \item Мигрировал 15+ сервисов с Docker Swarm на Kubernetes, повысил отказоустойчивость на 20\% и скорость работы на 45\%
        \item Разработал модель для автоматического обнаружения персональных данных (PII), снизил трудозатраты на проверку данных на 90\%
    \end{itemize}
    \vspace{-10pt}
    Стек: Python, GigaChat2, Qwen3, LangChain, FAISS, Kubernetes, RabbitMQ, MCP
    \vspace{-5pt}

    \item \textbf{ML Engineer / Team Lead} \hfill \textit{09.2021 -- 03.2024} \\
    \textit{\textbf{Proscom}, SalaryScan - ML-платформа для расчета заработной платы, Москва, Россия} \\
    \vspace{-10pt}
    \begin{itemize}
        \item Руководил командой из 5 человек, построил процесс разработки (Jira+Confluence), внедрил Agile, сократил time-to-production на 25\%
        \item Разработал систему парсинга резюме с открытых источников, масштабировал с 1.7K до 10K+/день через Hadoop+NiFi, снизил ручную разметку на 80\%
        \item Внедрил модель NER для автоматического выявления ключевых навыков и квалификаций кандидатов, улучшил MAE с 12.4K до 10.2K руб. в прогнозе зарплат
        \item Разработал рекомендательную систему для подбора кандидатов, увеличил конверсию продаж на 37\%
        \item Организовал 4 внутренних технических митапа (>60 участников), конверсия в найм 10\%
    \end{itemize}
    \vspace{-10pt}
    Стек: Python, XGBoost, CatBoost, Hadoop, NiFi, PostgreSQL, Docker
    \vspace{-17pt}
\end{itemize}

%-----------EDUCATION-----------
\section{Образование}

\begin{itemize}[leftmargin=0.15in, label={}]
    \item \textbf{НИУ ВШЭ}, Москва \hfill \textit{2025 -- настоящее время} \\
    Магистратура, Бизнес-информатика, Управление цифровым продуктом
    \vspace{-10pt}
    
    \item \textbf{РТУ МИРЕА}, Москва \hfill \textit{2021 -- 2024} \\
    Бакалавриат, Прикладная математика (красный диплом, GPA: 5.0/5.0)
    \vspace{-17pt}
\end{itemize}

\end{document}
`